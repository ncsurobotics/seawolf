
\newpage
\section{Seawolf Build System} \label{build}
The build system for the software stack consists of numerous makefiles which
provide information necessary for build automation by the GNU Make
utility. While the GNU version of make is targeted, it is not unlikely that
other versions should work as well.

\subsection{Using the Build System} \label{buildusing}
In the directories \folder{applications}, \folder{doc}, \folder{serial},
and \folder{libseawolf} are top-level makefiles (\file{Makefile}) allowing,
respectively, for the applications, this documentation, the serial drivers,
and \libseawolf{} itself to be built. To build an individual component, navigate
to the directory (e.g. \folder{applications}) and execute \texttt{make}. This
should start the build process. The applications and serial drivers both
use \libseawolf{} so it is necessary that \libseawolf{} be built before either
the applications or serial drivers be built. The documentation may be built at
any time. In addition to the default target, each of these makefiles support a
target to remove all generated files (execute \texttt{make clean}).

\subsection{Build Dependencies} \label{builddepends}
At the present, \libseawolf{} requires a Linux host as both a build and run
environment. In addition to a standard GCC build chain and libc, the MySQL
client libraries and headers should be installed (Debian
package \texttt{libmysqlclient15-dev}).

