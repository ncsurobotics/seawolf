
\newpage
\section{Protocols and File Formats} \label{formats}
\subsection{Seawolf Configuration File} \label{formatsseawolfconf}
This file is a single configuration file for configuring all components of
\libseawolf{}. The file is formatted as follows. Comments are ignored by the
parser and extend from a '\#' to the end of the current line. Whitespace is
ignored expect for newlines which separate records. Records are in the format
\bracket{option} = \bracket{value}. The following example configuration file
outlines all options and their possible values.
\begin{lstlisting}[caption=Example Seawolf configuration file]
# Seawolf configuration file

# MySQL host
SeaSQL_hostname = localhost

# MySQL user name
SeaSQL_username = root

# MySQL database
SeaSQL_database = test

# MySQL password
#SeaSQL_password = 

# Use a server running on localhost for notify IO
Notify_method = net
Notify_server = localhost

# Replicate all logging messages to standard output
Logging_replicateStdio = true

# Debug logging level. One of debug, info, normal,
# warning, or critical
Logging_threshold = debug
\end{lstlisting}

This file is located at \file{conf/seawolf.conf} under the top level software
directory. All existing applications look for it in this location and any new
code should reference this location.

\subsection{ArdComm Serial Protocol} \label{formatsardcomm}
\textit{This section yet to be compeleted}

