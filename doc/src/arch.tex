
\newpage
\section{Seawolf Component Architecture} \label{architecture}
\subsection{Components} \label{archcomp}
\subsubsection{\libseawolf{}} \label{libarch}
Seawolf applications are built around the Seawolf library known as
\libseawolf{}. This library forms the foundation for any application being
developed as part of Seawolf and provides functionalities to allow communication
between Seawolf applications, provide persistant, shared storage of data through
the SeaSQL database component and hardware interfacing through the serial
component. \libseawolf{} contains more components in addition to these mentioned
that provide additional functionality to Seawolf application developers to
utilize.

\subsubsection{Applications} \label{apparch}
Any applications which fully utilizes \libseawolf{} are considered ``Seawolf
Applications'' and are automatically placed into active communication with all
other Seawolf applications. With the exception of the hub server, all
applications which run as part of Seawolf should be Seawolf applications. This
includes all serial applications which are simply a special class of Seawolf
applications.

\subsubsection{Serial Management} \label{serialarch}
Seawolf's software must be able to integrate with a wide array of hardware
devices and providing as much isolation and abstraction between these hardware
devices and Seawolf applications as possible is highly desirable. Seawolf's
serial management applications were written to do this. To provide this
abstraction, Seawolf's serial applications handle all direct communication with
serial devices on the behalf of Seawolf applications. Seawolf applications can
indirectly communicate with devices by writing and reading from special
variables managed by SeaSQL. These variables act as a single channel of
communication with external devices. For example, there is a serial application
which communicates with a thruster control board. Seawolf application developers
can make calls to \func{SeaSQL\_setThruster} to update the value of what this
thruster should run at. The serial thruster application will look for this
variable to be updated and then communicate these updates to the thruster
control board. The serial applications include one \texttt{serialapp}
application which is responsible for identifying and spawning handlers for
serial devices. The rest of the serial applications are responsible for managing
particular serial devices such as thrusters, depth sensors, or altimeters.

\subsection{Architecture} \label{archarch}
Seawolf's software forms a multi application architecture where applications
perform single tasks and the combination of these individual applications create
higher level functionality. These applications communicate with one another
through a database exposed through SeaSQL and through a event notification
system provided by Notify. These components use network communication even when
applications are running on the same physical system. This choice of
communication medium means that it is trivial to grow the computational power of
Seawolf from a single system to multiple networked machines which are just as
easily able to communicate. To provide this communication two auxilliary
applications must be run. First is a MySQL database to provide data
persistence. The second requirement is that a ``hub'' server is running. This
server is distributed with \libseawolf{} and is responsibly for distributing
message sent through the Notify system.
