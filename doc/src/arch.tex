
\newpage
\section{Seawolf Component Architecture} \label{architecture}
\subsection{Components} \label{archcomp}
\subsubsection{\libseawolf{}} \label{libarch}
Seawolf applications are built around the Seawolf library known as
\libseawolf{}. This library forms the foundation for any application being
developed as part of Seawolf and provides functionalities to allow communication
between Seawolf applications, provide persistant, shared storage of data through
the SeaSQL database component and hardware interfacing through the serial
component. \libseawolf{} contains more components in addition to these mentioned
that provide additional functionality to Seawolf application developers to
utilize.

\subsubsection{Applications} \label{apparch}
\libseawolf{} does nothing by itself, but it used by 
applications which provide all control and logic for the craft. These
applications are simply refered to as ``Seawolf Applications'' and form the bulk
of the code base. Not included with the applications are the serial drivers
which are considered separate, and the hub which is considered to be part
of \libseawolf{}.

\subsubsection{Serial Management} \label{serialarch}
Seawolf is responsible for managing many IO devices and nearly all of these
operate over serial connections. Serial management is handled by individual
user-space drivers running for each IO device and a serial management
application (\texttt{serialapp}) is responsible for associating drivers with
serial devices through a process which automatically detects the device type on
each serial port. Each serial driver, once running, communicates with
applications through \libseawolf{} to provide abstracted access to the hardware
devices.

\subsection{Architecture} \label{archarch}
Seawolf's software forms a multi application architecture where applications
perform single tasks and the combination of these individual applications create
higher level functionality. These applications communicate with one another
through a database exposed through SeaSQL and through a event notification
system provided by Notify. These components use network communication even when
applications are running on the same physical system. This choice of
communication medium means that it is trivial to grow the computational power of
Seawolf from a single system to multiple networked machines which are just as
easily able to communicate. To provide this communication two auxilliary
applications must be run. First is a MySQL database to provide data
persistence. The second requirement is that a ``hub'' server is running. This
server is distributed with \libseawolf{} and is responsibly for distributing
message sent through the Notify system.
